\documentclass[11pt]{beamer}
\usetheme{JuanLesPins}
\usepackage{xspace}
\usepackage[utf8]{inputenc}
\usepackage[english]{babel}
\usepackage{amsmath}
\usepackage{listings, xcolor}
\usepackage{amsfonts}
\usepackage{amssymb}
\newcommand{\idawi}[1]{\textsc{Idawi}\xspace}

\usepackage{graphicx}
\author{Luc Hogie}
\title{\idawi - a parallel/distributed librairy tailored to Research in IOT/fog/edge/distributed computing}
%\setbeamercovered{transparent} 
%\setbeamertemplate{navigation symbols}{} 
%\logo{} 
\institute{Cnrs/Inria/Université Côte d'Azur} 
%\date{} 
%\subject{} 
\begin{document}

\begin{frame}
\titlepage
\end{frame}

%\begin{frame}
%\tableofcontents
%\end{frame}


\begin{frame}
\frametitle{}
\idawi, a new middleware for parallel/distributed computing for:
\begin{itemize}
	\item multi-hop
	\item decentralized
	\item unreliable
	\item mobile
\end{itemize}
Implementation: \url{https://github.com/lhogie/idawi}
\end{frame}


\begin{frame}
\frametitle{What is this all about?}
This is about \idawi, a new library for parallel/distributed computing.
\idawi is designed to:
\begin{itemize}
	\item meet the needs of our on-going/future R\&D projects:
		\begin{itemize}
			\item emulating IOT/decentralized networks (Map/Liquori)
			\item providing infrastructure services to Gemoc (Deantoni)
			\item computing very large graphs (Coati)
		\end{itemize}
	\item be useful to other labs (Open Source license)
	\item stand as a step forward existing tools
\end{itemize}
Implementation: \url{https://github.com/lhogie/idawi}
\end{frame}


\begin{frame}
\frametitle{Lastest product of a software suite?}
\idawi benefits from the experience we gained in past devs:
\begin{description}
	\item[BigGrph] set of modules dedicated to distributed graph computing. Supported by Inria. Now discontinued.
	\item[JMR] disk-based distributed batch processor.
	\item[JMaxGraph] multi-thread graph library for  processing large graphs.
	\item[MPI4Lectures]multi-thread message-passing library dedicated to teaching at University of Luxembourg
\end{description}
Also, \idawi learns a lot from ProActive,  a distributed implementation of the Fractal specification  developed by another team in our lab.
\end{frame}


\begin{frame}
\frametitle{In a few words...}
\idawi enables its user to execute software components on networked computers. These components have message-oriented communication facilities that is embeds built-in and user-defined services   will interact with each other by sending/receives messages. Also they expose . Doing this, they  distributed application made of services within the components.
\end{frame}



\begin{frame}
\frametitle{Components: what they \emph{are}}
There is no global consensus on what a component is. In \idawi, a component is an object augmented with the following features:
\begin{itemize}
	\item it organizes, along with other components, in a multi-graph
	\item it is able send/receive \emph{messages} to/from other components, using a variety of transport protocols
	\item it forwards incoming messages
	\item it exposes \emph{services} which extend its  functionality
\end{itemize}
\end{frame}


\begin{frame}
\frametitle{Components: what they \emph{are used for}}
\idawi components can be used to several purposes:
\begin{itemize}
	\item exposing computational resources in a cluster
	\item simulating complex systems by representing domain-specific entities and their internals
	\item providing services in a distributed application
\end{itemize}
\end{frame}


\begin{frame}
\frametitle{Built-in services}
Ranging from simplest to more sophisticated:
\begin{description}
	\item[exit] shuts down the entire distributed system
	\item[ping pong] emulates the ubiquitous ping command
	\item[error log] archives and disseminates internal errors
	\item[service lifecycle] starts/stops services remotely
	\item[routing] guides messages traveling across the  graph
\end{description}
\end{frame}


\begin{frame}
\frametitle{Built-in services}
Ranging from simplest to more sophisticated:
\begin{description}
	\item[bencher] provides performance information about the host computer
	\item[maps]  constructs a graph representing the topology of the network
	\item[deployment] enables new components to be started 
	\item[time series database] stores and serves numerical time-based information on user-defined metrics
	\item[publish-subscribe] notifies subscriber components of new publications on particular topics
\end{description}
\end{frame}



\begin{frame}
\frametitle{Service}
A service exposes to other components functionality for a particular concern. 
\begin{itemize}
	\item it proposes a high-level API for communication
	\item it defines a set of \emph{operations}
\end{itemize}
\end{frame}



\begin{frame}
\frametitle{operations are what perform user-defined computations}
\begin{itemize}
	\item is a piece of sequential code
	\item triggered by a \emph{message}
	\item read input from a message list
\end{itemize}
At runtime, an operation can:
\begin{itemize}
	\item do anything, including sending messages to any other component/service
	\item invoke operations on any other component/service (service composition)
\end{itemize}
\end{frame}

\begin{frame}
\frametitle{Parallel execution}
\begin{itemize}
	\item unlike active objects, components have no threads
	\item $#threads = f(#cores)$
	\item lock-free by default
\end{itemize}
\end{frame}




\begin{frame}
\frametitle{Message}
\begin{itemize}
	\item carries a content (that can be anything)
	\item to a set of components defines in a "to:"  address
	\item has an optional "reply-to:" (that is automatically fed in the case of a synchronous emission)
\end{itemize}
An address consists of:
\begin{itemize}
	\item an optional set of recipients components (unicast if $|R|=1$, multicast if $|R|>1$, broadcast if $R=null$)
	\item a recipient service that is expected on all recipients components
	\item a recipient operation
\end{itemize}
\end{frame}




\begin{frame}
\frametitle{Communication}
\begin{itemize}
	\item message emission is asynchronous
	\item no guaranty of delivery
	\item if a message has "reply-to" information, the sender obtains a queue that will store response messages.
\end{itemize}
Queues enable:
\begin{itemize}
	\item asynchronous on-the-fly  processing of incoming messages
	\item synchronous collect and classification
\end{itemize}
\end{frame}


\begin{frame}
\frametitle{Communication protocols}
\idawi currently support the following transport protocols:
\begin{itemize}
	\item TCP: ensures reception
	\item UDP: is quick
	\item IPC: connect components in child/parent processes
		\begin{itemize}
			\item locally, mostly useful for tests
			\item remotely through SSH, across NATs and firewalls
		\end{itemize}
	\item intra-process: method calls enables large iocal simulations
\end{itemize}
\end{frame}


\begin{frame}
\frametitle{Deployment of components}
New components can be deployed anywhere a SSH connection is possible (and rsync is available). When you 
\begin{enumerate}
	\item determines shared file systems among computers
	\item update binaries, which includes:
		\begin{enumerate}
			\item incremental update of Java  bytecode (using rsync)
			\item installs the right JVM if necessary
		\end{enumerate}
	\item executes a JVM  and starts a component in it
\end{enumerate}
The parent component initially communicate with the new component through the standard I/O streams of the SSH local process. If the new component is declared to be autonomous, it remains alive even if the I/O streams get closed.
\end{frame}

 

\begin{frame}
\frametitle{Interoperable with external tools using REST}
\begin{itemize}
	\item the REST service launches a HTTP server
	\item which serve JSON documents provided by specific REST operations in services
	\item the REST interface gives access to the  component system as a whole, regardless of which component exposes it
	\item executing a REST operation is done via the URL:  \resizebox{.9\hsize}{!}{\url{http://host:port/component/service/operation/parms1,parm2,...,parm3}}
\end{itemize}
\end{frame}


\begin{frame}
\frametitle{Conclusion: most notable features}
\begin{enumerate}
	\item multihop mobile overlay network of components
	\item deployment of new remote/local component  
	\item synchronous and asynchronous communications
	\item support for unicast/multicast/broadcast
	\item reactive message programming and stream processing
	\item massively parallel computations
	\item multi-protocol, REST/JSON interface
	\item many base services for platform management and demo
\end{enumerate}
\end{frame}


\begin{frame}
\frametitle{Adding/deleting operations}
operations can be declared by:
\begin{description}
	\item[calling Service.registerNewOperation()] allows adding/deleting new operations are runtime
	\item[adding annotated methods] improves readability by isolating the code, allowing the specification of a return and parameters type. Such an operation can be unregistered but its implementation code remains, and can still be invoked from within its class.
\end{description}
\end{frame}


\begin{frame}[containsverbatim]
\frametitle{Sending a message}
operations can be declared by:
\begin{lstlisting}
To to = new To(
MyResultType r = (MyResultType) send(content,
			Set.of(myRemoteComponent),
			myService, myoperation)
	.setTimeout(5)
	.collect()
	.ensureResults(1)
	.first().content;
\end{lstlisting}
\end{frame}



\begin{frame}
\frametitle{Conclusion}
\end{frame}

\end{document}
